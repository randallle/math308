\documentclass[12pt, letterpaper]{article}
\usepackage[utf8]{inputenc}
\setlength{\parindent}{0pt}

\usepackage[margin=1in]{geometry}
\usepackage{amsmath}
\usepackage[none]{hyphenat}
\usepackage{amssymb}
\usepackage{amsfonts}
\usepackage{amstext}
\usepackage{amsthm}

\usepackage{fancyhdr}
\pagestyle{fancy}
\fancyhf{}
\lhead{MATH 308}
\rhead{Week 3: Friday, January 28, 2022}

\begin{document}
\textbf{Hypothesis testing using the Z-test} \\
We want to compare measurements in a certain group to those of a population\\
$$\overline{x}=\frac{x_1+x_2+...+x_n}{n}$$ is a value of a random variable that is approximately normally distributed with sample mean $\overline{x}$ and sample standard deviation, $s$ \\
\begin{center}
Note: $\overline{x}$ is the estimate for the group mean, $\mu$
\end{center}
$$P_H(|Z|\geq \frac{\overline{x}-\mu_0}{\sigma_0})= P(|Z|\geq \frac{\overline{x}-\mu_0}{\sigma_0}|H_0) = \text{p-value} $$\\
When the \underline{p-value} is less than 5\%, we reject the null hypothesis, $H_0$\\

\textbf{Remarks:}
\begin{enumerate}
    \item If the population SD, $\sigma_0$, is unknown, we use the unbiased sample standard deviation
    $$s=\sqrt{\frac{1}{n-1}\sum_{i=1}^{n} (x-\overline{x})^2} $$
    \item If n $<30$, we use the T-test
    \item $P_H(|Z|\geq \frac{\overline{x}-\mu_0}{\sigma_0})$ is a two-tail test
\end{enumerate}
\textbf{Example: }A course with 900 students randomly broken down into 30-student discussion sections. The final exam average is 63 and the standard deviation is 20. In one section, the average is 55. What is the probability of obtaining a 55 in this group if students were given the same lecture?\\\\
\textbf{Solution:}\\
The sample of 30 scores can be combined as independent measurements following the same sample process.\\\\
According to the Central Limit Theorem, $\overline{x}=55$ is a value coming from a random variable $\overline{x}$ that is approximately normally distributed with mean $\mu$ and SD$=\frac{\sigma}{\sqrt{n}}=\frac{20}{\sqrt{30}}$\\\\
Null hypothesis, $H_0$: $\overline{X}~\mathcal{N}(\mu_0=63, \sigma_0=\frac{20}{\sqrt{30}})$\\\\
If $H_0$ is true, then $z=\frac{\overline{x}-\mu_0}{\sigma_0}=\frac{55-63}{\frac{20}{\sqrt{30}}}=-2.2$

What is P(

% The test is used when the 2 samples are dependent. That is, when the values in the two samples are pairwise matched (i.e. clinical trials)
% $$\text{Sample}_A=(X_{A1}, X_{A2}, X_{A3}, ..., X_{An})$$
% $$\text{Sample}_B=(X_{B1}, X_{B2}, X_{B3}, ..., X_{Bn})$$



% $$ t=\frac{\overline{x_D}-\mu_0}{\frac{S_D}{\sqrt{n}}} \text{ where } x_D = (x_{A1}-x_{B1}, x_{A2}-x_{B2}, ..., x_{An}-x_{Bn})$$ \\

% \textbf{Example:} A group of 20 sports science students are randomly selected to investigate whether a 12-week plyometric-training program improves their standing long-jump performance.\\\\The following are the test results:

% $$\overline{x}_D = \frac{1}{n}\sum_{k=1}^{n} (x_{Ak}-x_{Bk}) = 0.034$$
% $$\overline{s}_D = \sqrt{\frac{1}{n-1}\sum_{k=1}^{n} (x_{Dk}-\overline{x}_D)^2} = 0.03185$$

% Is the difference statistically significant?\\\\
% \textbf{Solution:}\\
% Let's use a paired 2-sample t-test. We're assuming the data were not too skewed ($n\leq20$)\\\\
% Null hypothesis, $H_0$: the distribution for the mean of the sample difference is more or less $\mathcal{N}(0,1)$

% $$t=\frac{\overline{x}_D-0}{\frac{s_D}{\sqrt{n}}} = \frac{0.034-0}{\frac{0.03185}{\sqrt{20}}}\approx 4.77$$
% $$\text{df}=20-1=19$$\
% $$P_{H_0}(T\geq4.77)\approx0\%\Rightarrow\text{ Reject }H_0$$ 

% \pagebreak
% \textbf{Preview of Estimators}
% \begin{itemize}
%     \item $\mu$ is the mean in a population
%     \item $\hat{\mu}$ is a random variable whose values should estimate $\mu$
%     \item \textbf{Example of estimator:} $\hat{\mu}$ where \{all n-samples\} $\Rightarrow \mathbb{R}$ 
%     \item \textbf{Definition:} an estimator is unbiased if $E(\hat{\mu})=\mu$
%     \item Trivial example: 
% \end{itemize}







% \hat{\delta}: {all "nice" n-samples}



\end{document}
