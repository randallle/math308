\documentclass[12pt, letterpaper]{article}
\usepackage[utf8]{inputenc}
\setlength{\parindent}{0pt}
% \renewcommand{\arraystretch}{1.5}

\usepackage[margin=1in]{geometry}
\usepackage{amsmath}
\usepackage[none]{hyphenat}
\usepackage{amssymb}
\usepackage{amsfonts}
\usepackage{amstext}
\usepackage{amsthm}
\usepackage{multicol}
% \usepackage{parskip}

\usepackage{fancyhdr}
\pagestyle{fancy}
\fancyhf{}
\lhead{MATH 308}
\rhead{Week 4: Friday, February 4, 2022}

\begin{document}
    \textbf{T-test:} for sample sizes $\leq$ 30-50 \bigskip

    Validity conditions
    \begin{enumerate}
        \item Simple random sample
        \item Sample distribution is symmetric about the sample mean
        \begin{itemize}
            \item Or size $\geq$ 20 and sample distribution is not strongly skewed
        \end{itemize}
    \end{enumerate}\bigskip

    \textbf{One sample mean test}

    \begin{multicols}{2}
        \vfill\null$$\boxed{t=\frac{\overline{x}-\mu_0}{\frac{s}{\sqrt{n}}}}$$
        \vfill\null\columnbreak
        Where
        \begin{itemize}
            \item $\overline{x}$ is the sample mean
            \item $\mu_0$ is the population mean
            \item $s=\sqrt{\frac{1}{n-1}\sum_{i=1}^{n} (x_i-\overline{x})^2}$\\
            is the unbiased standard deviation
        \end{itemize}
    \end{multicols}

    \textbf{Null hypothesis, $H_0$}: sample mean distribution and population means distribution are the same, with mean $\mu_0$ and standard deviation $s$\bigskip

    \textbf{Degrees of freedom:} $df=n-1$ \bigskip\bigskip

    \textbf{Example:} We have a sample of 5 measurements, $\{78, 83, 68, 72, 88\}, \text{ and } \mu_0=70$ \bigskip

    \textbf{Solution:} $\overline{x}=\frac{78+...+88}{5}=\boxed{77.8}$
    $$s=\sqrt{\frac{1}{n-1}\sum_ {i=1}^{n} (x_i-\overline{x})^2}= \sqrt{\frac{1}{4}[(78-77.8)^2+...+(88-77.8)^2]} = \boxed{8.07} $$

    $$df=5-1=\boxed{6}$$

    $$t=\frac{\overline{x}-\mu_0}{\frac{s}{\sqrt{n}}} = \frac{77-8-70}{\frac{8.07}{\sqrt{5}}} = \boxed{2.2}$$

    \begin{align*}
        \text{\underline{Two-tail test:} } &P_{H_0}(|T|\geq2.2)\leq0.05\Rightarrow\boxed{\text{Reject } H_0} \\
        \text{\underline{One-tail test:} } &P_{H_0}(T\geq2.2)\leq0.05
    \end{align*}
    

    


\end{document}

% \begin{document}
% \textbf{Bias and Unbiased Estimators}

% \textbf{Definition:} An estimator is a rule, often expressed as a formula, that tells us how to calculate the value of an estimate for a parameter based on the value in a sample

% \textbf{Example: }Sample mean is a (point) estimator, $\hat{\mu}$
% $$\hat{\mu}: \{\text{n-sample}\} \rightarrow \mathbb{R}$$
% $$(y_1, y_2, ..., y_n) \rightarrow \frac{y_1 + y_2 + ... +y_n}{n}$$

% \textbf{Remark:} The estimator $\hat{\mu}$ can be seen as a random variable
% $$\hat{\mu}=\overline{Y}=\frac{Y_1 + Y_2 + ... + Y_n}{n}$$

% The hope is that $E(\hat{\mu})=\mu$ \bigskip

% \textbf{Definition:} Let $\hat{\theta}$ be a \underline{point estimator} of a parameter $\theta$. $\hat{\theta}$ is said to be an unbiased estimator if $E(\hat{\theta})=\theta$. Otherwise, it is biased. \bigskip

% \textbf{Definition:} 
% \begin{enumerate}
%     \item The \textbf{bias} of $\hat{\theta}$ is \framebox[1.1\width]{$B(\hat{\theta})=E(\hat{\theta})-\theta$}
%     \item The \textbf{mean square error of $\hat{\theta}$} is \framebox[1.1\width]{$\text{MSE}(\hat{\theta})=E[(\hat{\theta}-\theta)^2]$}
% \end{enumerate}
% \bigskip
% \textbf{Theorem:} $\text{MSE}(\hat{\theta}) = V(\hat{\theta}) + B(\hat{\theta})^2$ \bigskip

% \textbf{Proof:}\
% By definition $\text{MSE}(\hat{\theta})=E[(\hat{\theta}-\theta)^2]$. Therefore,
% \begin{align*}
%     \text{MSE}(\hat{\theta})&=E[\hat{\theta}^2-2\theta\hat{\theta}+\theta^2] \\
%     &=E[\hat{\theta}^2]-2\theta E[\hat{\theta}] + \theta^2 \\
%     &=E[\hat{\theta}^2]-2\theta E[\hat{\theta}] + \theta^2    
% \end{align*}

% \textbf{Common unbiased estimators}\\
% We're assuming that the sampling method produces random, independent values in the sample

% \begin{center}
% \begin{tabular}{|c |c |c |c |c |}
%     \hline
%     Target parameter & Sample size & Estimator ($\hat{\theta}$) & $E(\hat{\theta})$ & Standard error ($\sigma_{\hat{\theta}}$) \\ 
%     \hline
%     $\theta$ & $n$ & $\hat{\mu}=\overline{x}$ & $\mu$ &$\frac{\sigma}{\sqrt{n}}$ \\  
%     \hline
%     $p$ & $n$ & $\hat{p}=\frac{\sum \text{successes}}{n} $ & $p$ & $\sqrt{\frac{pq}{n}}$ \\
%     \hline
%     $\mu_1-\mu_2$ & $n_1 \text{ and }n_2$ & $\widehat{\mu_1 -\mu_2}=\hat{\mu_1}-\hat{\mu_2}$ & $\hat{\mu_1}-\hat{\mu_2}$ & $\sqrt{\frac{\sigma_1^2}{n_1} + \frac{\sigma_2^2}{n_2}}$ \\
%     \hline
%     $p_1-p_2$ & $n_1 \text{ and }n_2$ & $\widehat{p_1-p_2}=\hat{p_1}-\hat{p_2}$ & $p_1-p_2$ & $\sqrt{\frac{p_1q_1}{n_1}+\frac{p_2q_2}{n_2}}$ \\
%     \hline
% \end{tabular}
% \end{center}

% Less trivial example of unbiased estimator
% $$\hat{\sigma}^2 = \frac{1}{n-1}\sum_{i=1}^{n} (y_i-\overline{y})^2  $$ \\\\

% \textbf{Proof:}
% $$E(\sigma^2) = E(\frac{1}{n-1}\sum_{i=1}^{n} (y_i-\overline{y})^2)$$
% $$=\frac{1}{n-1}\sum_ {i=1}^{n} E[(Y_i-\overline{Y})^2]$$

% Here: the $Y_i$ are independent random variables with $E(Y_i)=\mu$ and $V(Y_i)=\sigma^2$. Therefore, $E(Y_iY_j)=E(Y_i)E(Y_j) = \mu\mu$ if $i \neq j$



% \end{document}
